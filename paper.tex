%!BIB program=biber

\documentclass{article} %类型为文章
\usepackage[UTF8]{ctex} %中文编码宏
\usepackage[hidelinks]{hyperref} %超链接宏
\usepackage{geometry} %页面控制宏
\usepackage{fancyhdr} %页眉页脚宏
\usepackage{lastpage} %总计页的宏
\usepackage{color} %颜色控制宏
\usepackage{graphicx} %图片插入宏
\usepackage{subfigure} %子图插入宏
\usepackage{diagbox} %表格斜线宏
\usepackage{multirow} %纵向合并宏
\usepackage{makecell} %表格换行宏
\usepackage{amsmath} %公式插入宏
\usepackage{cases}
\usepackage{unicode-math} %公式样式宏
\usepackage{algorithm2e} %伪代码宏
\usepackage{gbt7714} %国标引用宏
\usepackage{url} %网页链接宏
\usepackage{doi} %doi号宏
\renewcommand{\vec}[1]{\boldsymbol{#1}} %设置向量样式

\usepackage{listings}
\usepackage{color}
\definecolor{dkgreen}{rgb}{0,0.6,0}
\definecolor{gray}{rgb}{0.5,0.5,0.5}
\definecolor{mauve}{rgb}{0.58,0,0.82}
\lstset{frame=tb,
  language=Python,
  aboveskip=3mm,
  belowskip=3mm,
  showstringspaces=false,
  columns=flexible,
  basicstyle={\small\ttfamily},
  numbers=left,%设置行号位置none不显示行号
  %numberstyle=\tiny\courier, %设置行号大小
  numberstyle=\tiny\color{gray},
  keywordstyle=\color{blue},
  commentstyle=\color{dkgreen},
  stringstyle=\color{mauve},
  breaklines=true,
  breakatwhitespace=true,
  escapeinside=``,%逃逸字符(1左面的键),用于显示中文例如在代码中`中文...`
  tabsize=4,
  extendedchars=false %解决代码跨页时,章节标题,页眉等汉字不显示的问题
}

\geometry{a4paper,left=2cm,right=2cm,top=2cm,bottom=2cm,headsep=0.5cm,footskip=1cm} %设置页边距和页眉页脚距离
\pagestyle{fancy} %设置页面样式
\fancyhf{} %开启页眉页脚
\lhead{欧纪阳 2019141220016} %设置左侧页眉为作者
\rhead{Sichuan University 四川大学} %设置右侧页眉为机构
\cfoot{第\thepage 页 \quad 共 \pageref{LastPage} 页} %设置居中页脚为页码

\linespread{1.2} %行距
\setlength{\parskip}{0.5em} %段落间距
\setlength{\parindent}{2em} %缩进距离

\setmathfont{Cambria Math} %设置数学公式样式
\bibliographystyle{gbt7714-numerical} %设置参考文献样式

\title{电动力学数值实验报告} %设置标题
\author{欧纪阳 2019141220016\\ \textit{College of Physics, Sichuan University, Chengdu 610064, China}} %设置作者
\date{\today} %设置日期

\begin{document}
\maketitle %插入标题
\begin{abstract} %插入摘要和关键字
    \quad 本文报告了电荷电场之间相互计算的数值方法,以及推导单带电粒子在空间中运动的数值方法。利用迭代求解有限空间中的泊松方程,可以在给定的电荷分布和边界条件下求解电势与电场分布。对于电场推导电荷的方法为根据电场的散度为电荷分布,直接对利用三点中心差分法替代梯度算符,即可获得电荷分布。在单粒子在电磁场中运动的问题中,对时间进行积分,使用梯形法则的数值办法,即可轻松获得粒子的运动轨迹。本文所有的代码均采用Python进行编写。
\end{abstract}

\tableofcontents %插入目录
\thispagestyle{empty} %本页无页眉页码

\newpage

\part{电荷计算电场}
该部分内容较长,另附文章《多重网格-五点差分-共轭梯度法——求解二维泊松方程的一类边值问题:从电荷分布求解电场的数值方法》

\part{电场计算电荷}

\section{方法}
对于一个三阶可微函数$f(x)$,根据泰勒定理有,
\begin{align}
    f(x+h) & =f(x)+hf'(x)+\frac{h^2}{2}f''(x)+\frac{h^3}{6}f^{(3)}(c_1) \\
    f(x-h) & =f(x)-hf'(x)+\frac{h^2}{2}f''(x)-\frac{h^3}{6}f^{(3)}(c_2)
\end{align}
其中$x-h<c_2<x<c_1<x+h$,最后一项为误差项,两式相减可以得到\textbf{一阶导数的三点中心差分公式},
\begin{align}
    f'(x)=\frac{f(x+h)-f(x-h)}{2h}-\frac{h^2}{6}f^{(3)}(c)=\frac{f(x+h)-f(x-h)}{2h}+O(x^3)
\end{align}
用该公式在离散情况下对$\vec{E}=-\nabla \varphi$做近似可得,
\begin{align}
    \vec{E}_{i,j}=\left(\frac{\varphi_{i+1,j}-\varphi_{i-1,j}}{2h_x},\frac{\varphi_{i,j+1}-\varphi_{i,j-1}}{2h_y}\right)
\end{align}
其中$\left\{\varphi_{i,j}|0\leq i\leq m-1,0\leq j\leq n-1\right\}$,$h_x,h_y$分别为$x$方向和$y$方向上的网格精度,利用以下代码可以在Python中实现此功能,其中Ex和Ey为网格矩阵(xx,yy)的函数,所有是空间位置函数的变量(例如Ex)的行标号用于表示$y$轴坐标,列标号用于表示$x$轴坐标。
\begin{lstlisting}
import numpy as np

d = np.array([1e-3, 1e-3])  # [y,x]
x = np.arange(xMin, xMax+d[1], d[1])
y = np.arange(yMin, yMax+d[0], d[0])

xx, yy = np.meshgrid(x, y)
Ex, Ey = Function(xx, yy)
Exy = np.stack((Ey, Ex), axis=2)
rho1 = (Exy[1:-1, 2:, 1]-Exy[1:-1, :-2, 1])/2/d[1]*epsilon0
rho2 = (Exy[2:, 1:-1, 0]-Exy[:-2, 1:-1, 0])/2/d[0]*epsilon0
rho = rho1+rho2
\end{lstlisting}

\section{结果}
\subsection{正余弦分布电场}
给予电场一正余弦分布,
\begin{align}
    E=-\pi\cos(\pi x)\sin(\pi y)\vec{e_x}+\pi\sin(\pi x)\cos(\pi y)\vec{e_y}
\end{align}
根据高斯定理$\nabla \cdot E = \rho / \varepsilon$,设$\varepsilon=1$,则可以得到电荷分布的解析解,
\begin{align}
    \rho=2\pi^2 \sin(\pi x)\sin(\pi y)
\end{align}

利用Python编写上述的求解方法,对此问题进行求解,设置求解区域为$S=\left\{(x,y)|-1\leq x\leq 1,-1\leq y\leq 1\right\}$的正方形区域,在两个维度上的网格精度均为$10^{-3}$,即求解区域的网格大小为$2000\times 2000$。该模型的数值计算结果以及误差如图\ref{F1}所示,由误差图可以看出计算精度已经达到了四位有效数字,在结果中数值越大的区域其误差也越大,根据三点中心差分公式,其误差来源可以认为是被忽略掉了的二阶小量$O(x^2)\sim (h_x^2+h_y^2)\pi^3 \sim 10^{-5}$,粗略估计由方法所带来的误差与实际计算的误差相符。
\begin{figure}
    \begin{center}
        \includegraphics[width=17cm]{e2q.png}
    \end{center}
    \qquad 左图为数值计算结果,第一象限与第三象限分别有一个波峰,波谷存在于第二和第四象限内;右图为与解析解对照的前向误差图,用灰度表示误差的大小,颜色约浅表示误差越大,衍射越深表示误差越小。
    \caption{正余弦电场分布的数值计算结果及误差图}
    \label{F1}
\end{figure}

\subsection{点电荷对应的电场分布}
已知点电荷在真空中所产生的电场的表达式为,
\begin{align}
    \vec{E}=\frac{Q \vec{r}}{4\pi \epsilon r^3}=(\frac{Qx}{4\pi \epsilon r^3},\frac{Qy}{4\pi \epsilon r^3})
\end{align}
利用该方法对该电场分布进行求解,设置求解区域为$S=\left\{(x,y)|-0.9995\leq x\leq 0.9995,-0.9995\leq y\leq 0.9995\right\}$的正方形区域,在两个维度上的网格精度均为$10^{-3}$,这样设置求解区域的原因是为了避免网格落在原点上,导致原点附近无法求解,求解得到的电荷分布如图\ref{F1.1}所示。可以看到电荷密度几乎完全分布在原点附近,但是其他区域的计算结果中依然还是出现了数量级较小的电荷分布,是由于方法本身的系统误差所造成的,在局部放大的图像中,可以看到原点位置的电荷密度非常高,原点附近电荷密度变化幅度巨大。
\begin{figure}
    \begin{center}
        \includegraphics[width=17cm]{s.png}
    \end{center}
    \qquad 左图为点电荷对应的电场分布所求解得到的数值实验结果,右图为原点附近$\left\{(x,y)|-0.02\leq x\leq 0.02,-0.02\leq y\leq 0.02\right\}$的放大图。
    \caption{数值计算的点电荷分布}
    \label{F1.1}
\end{figure}

\part{单粒子在均匀电磁场中的运动}

\section{方法}
考虑带电量为$Q$,质量为$m$的带电粒子在电磁场中的运动,其在电磁场中的运动方程为,
\begin{align}
    m\vec{a}=q \vec{v} \times \vec{B}+q \vec{E}
\end{align}
粒子运动的位移可以表示为其运动速度的积分,利用\textbf{梯形法则}可以写作,

\begin{minipage}[c]{0.5\linewidth}
    \begin{align}
        \vec{r}=\int_{t0}^{t1} \vec{v} \, \mathrm{d}t&=\sum_i \frac{\mathrm{d}t}{2}[v(t_i)+v(t_{i+1})]\\
        &=\sum_i v(t_i)\mathrm{d}t +\frac{1}{2} v'(t_i) \mathrm{d}t^2\\
        &=\sum_i v_i \mathrm{d}t +\frac{1}{2} a_i \mathrm{d}t^2
    \end{align}
\end{minipage}
\begin{minipage}[c]{0.4\linewidth}
    \includegraphics[width=7cm]{e.pdf}
\end{minipage}

利用以下代码可以在python中实现此功能,所有的矢量的三个分量按照(x,y,z)的顺序排列,其中np.cross函数表示向量的叉乘。
\begin{lstlisting}
    import numpy as np
    
    E = np.array([0., 0., 0.])  # 电场强度
    B = np.array([0., 0., 1.])  # 磁感应强度
    Q = 1  # 电荷量
    m = 1  # 质量
    r0 = np.array([-1., 0., 0.])  # 初始位移
    v0 = np.array([0., 1., 0.])  # 初始速度
    dt = 1e-3  # 时间精度
    T = 20*np.pi  # 求解时长
    tt = np.arange(0, T+dt, dt)
    
    r = np.array([r0])
    v = np.array([v0])
    for t in tt[1:]:
        a = (Q*E+Q*np.cross(v, B))/m
        dr = v[-1]*dt+(a*dt**2)/2
        dv = a*dt
        r = np.row_stack((r, r[-1]+dr))  # 更新一个位置,并连接到位置数组的最后一行之后
        v = np.row_stack((v, v[-1]+dv))  # 更新一个速度,并连接到速度数组的最后一行之后
    \end{lstlisting}

\section{结果}

\subsection{静磁场}
首先考虑只存在静磁场,不存在静电场的情况,令$Q=1,m=1,B=(0,0,1),r_0=(-1,0,0),v_0=(0,1,0)$,时间步长$\mathrm{d}t=10^-3$,时长$T=2\pi$,进行数值求解,粒子的运动轨迹如图\ref{F2}(a)所示,可以看到粒子在绕坐标原点做圆周运动,该问题具有解析解,
\begin{align}
    \vec{r}=(-\cos t,\sin t)
\end{align}
用数值计算与解析解进行比较,画出求解的误差曲线如图\ref{F2}(b)所示,可以看到误差随着迭代的时间而变大,且误差小于两个数量级。如果继续延长求解的时间达到$T=20\pi$,求解的粒子运动轨迹如图\ref{F2}(c)所示,可以明显的看到粒子的运动轨迹向内偏移,其求解误差曲线如图\ref{F2}(d)所示,由于时间的延长,误差在没有修正的情况下继续增大,已经达到了两个数量级,这主要是由于在迭代过程中省略了三阶小量所造成的。
\begin{figure}
    \subfigure[$2\pi$时间内粒子的运动轨迹]{\includegraphics[width=8.2cm]{B1.pdf}}
    \subfigure[$2\pi$时间内的数值计算误差]{\includegraphics[width=8.2cm]{B1e.pdf}}
    \\
    \subfigure[$20\pi$时间内粒子的运动轨迹]{\includegraphics[width=8.2cm]{B2.pdf}}
    \subfigure[$20\pi$时间内的数值计算误差]{\includegraphics[width=8.2cm]{B2e.pdf}}
    \caption{只存在磁场时的数值计算结果}
    \label{F2}
\end{figure}

\subsection{静电磁场}
在同时存在静电场与静磁场的情况下,一般不存在解析解的情况,一些特例除外(例如电场与磁场方向相同,洛伦兹力与电场力刚好抵消等),下面的几幅图(\ref{F2.1})展现了几种电磁场分布下的同一带电粒子($Q=1,m=1$)的运动轨迹(初始位置均为$r_0=(0,0,0)$):
\begin{itemize}
    \item (a)$E=(0,0,1),B=(0,1,0),v_0=(1,0,0)$
    \item (b)$E=(1,0,0),B=(0,1,0),v_0=(0,1,0)$
    \item (c)$E=(0,0,1),B=(0,0,1),v_0=(1,0,0)$
    \item (d)$E=(0,0,1),B=(0,1,1),v_0=(1,0,0)$
\end{itemize}

\section{附录}
本文的所有代码已经公开发布于GitHub上\url{https://github.com/812610357/Electrodynamics},欢迎访问我的数据库标星收藏跟随关注。

\begin{figure}
    \subfigure[]{\includegraphics[width=8.2cm]{EB1.pdf}}
    \subfigure[]{\includegraphics[width=8.2cm]{EB2.pdf}}
    \\
    \subfigure[]{\includegraphics[width=8.2cm]{EB3.pdf}}
    \subfigure[]{\includegraphics[width=8.2cm]{EB4.pdf}}
    \caption{列举几种粒子的运动轨迹}
    \label{F2.1}
\end{figure}

\end{document}